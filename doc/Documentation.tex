\documentclass{article}

\usepackage{ifxetex}
\ifxetex
    % XeLaTeX
    \usepackage{polyglossia}
    %\setmainlanguage{english}
    \setdefaultlanguage{english}
    \setkeys{english}{variant=british}
    \setotherlanguage[spelling=new,babelshorthands=true]{german}
    \usepackage{fontspec}
    \usepackage[]{unicode-math}
    %\setmainfont{XITS}
    %\setmathfont{XITS Math}
\else
    % default: pdfLaTeX
	%\usepackage[pdftex]{graphicx}
    \usepackage[british]{babel}
    \usepackage[T1]{fontenc}
    %\usepackage{lmodern}
    \usepackage[adobe-utopia]{mathdesign}
    \usepackage[utf8]{inputenc}
    \usepackage[babel=true]{microtype}
	\selectlanguage{british}
\fi

\usepackage[
    backend=biber,
    %style=authoryear-icomp,
    style=authoryear-comp,
    %style=apa,
    sortlocale=auto,
    language=british,
    natbib=true,
    url=true, 
    doi=false,
    eprint=false,
    hyperref=true,
    firstinits=true,
    uniquename=init,
    maxcitenames=2,
    parentracker=true,
    backref=true,
]{biblatex}
\DeclareLanguageMapping{british}{british-apa}

\addbibresource{Bib/Primary-Refs.bib}

\usepackage{listings}
\lstdefinestyle{TeX}{
  breaklines=true,
  frame=simple,
  language=TeX,
}
\lstset{style=TeX}

\usepackage{tikz}
\usetikzlibrary{calc}
\usepackage{../src/tikz-pf}
\begin{document}

\section{TikZ-pf}

Some problem frames provided as examples to show the usage.

\subsection{Context Diagram}

\begin{figure}[htbp]
\centering
\begin{tikzpicture}
	%\draw[help lines,step=1cm,gray] (-2,-3) grid (12,3);

	\domain[x=0,y=0,type=machine]{m}{Monitor machine};
	\domain[x=0,y=2,type=designedDomain]{pr}{Periods \& Ranges};

	\domain[x=0,y=-2]{ms}{Medical staff};
	\domain[x=4.5,y=2]{ns}{Nurses' station};
	\domain[x=4.5,y=0]{fd}{Factors database};
	\domain[x=4.5,y=-2]{ad}{Analog devices};
	\domain[x=9,y=-2]{icu}{ICU patients};

	\connects[label position=left]{m}{pr}{a}
	\connects[label position=left]{m}{ms}{b}
	\connects{m}{ns}{c}
	\connects{m}{fd}{d}
	\connects[label position=below]{m}{ad}{e}
	\connects{ad}{icu}{f}


	\begin{scope}[align=flush left, text width=60mm, font=\small]
		\node at (9,1) {
		\begin{description}
		\itemsep0em
		\item a: Period, Range, PatientName, Factor
		\item b: EnterPeriond, EnterRange, EnterPatientName, EnterFactor
		\item c: Notify
		\item d: Factors
		\item e: RegisterValue
		\item f: FactorEvidence
		\end{description}
		};
	\end{scope}
\end{tikzpicture}


\caption{{Context Diagram}: Patient Monitoring System (cf.~\cite{Jackson:2001}) \label{fig:cd}}
\end{figure}
\lstinputlisting{Figures/cd.tex}

\subsection{Problem Frames}

\subsubsection{Required Behaviour}
\begin{figure}[htbp]
\centering
\begin{tikzpicture}

	\domain[x=0,y=0,type=machine]{m}{Control machine};
	\domain[x=5,y=0,type=casualDomain]{cd}{Controlled Domain};

	\connects[yshift=-12pt]{m}{cd}{CM!C1\\ CD!C2}

	\requirement[x=10,y=0]{r}{Required behaviour};

	\constrains{r}{cd}{C3}

\end{tikzpicture}


\caption{{Required Behaviour} {Problem Frame} (cf.~\cite{Jackson:2001}) \label{fig:rb}}
\end{figure}
\lstinputlisting{Figures/rb.tex}

\subsubsection{Commanded Behaviour}
\begin{figure}[htbp]
\centering
\begin{tikzpicture}

	\domain[x=0,y=0,type=machine]{m}{Control machine};
	\domain[x=5,y=.7,type=casualDomain]{cd}{Controlled Domain};
	\domain[x=5,y=-.7,type=biddableDomain]{o}{Operator};

	\connects{m}{cd}{CM!C1\\ CD!C2}
	\connects[label position=below]{m}{o}{O!E4}

	\requirement[x=10,y=0]{r}{Commanded behaviour};

	\constrains{r}{cd}{C3}
	\refers[label position=below]{r}{o}{E4}

\end{tikzpicture}


\caption{{Commanded Behaviour} {Problem Frame} (cf.~\cite{Jackson:2001}) \label{fig:cb}}
\end{figure}
\lstinputlisting{Figures/cb.tex}

\subsubsection{Commanded Information}
\begin{figure}[htbp]
\centering
\begin{tikzpicture}

	\domain[x=0,y=0,type=machine]{m}{Answering machine};
	\domain[x=5,y=1.3,type=casualDomain]{rw}{Real world};
	\domain[x=5,y=0,type=casualDomain]{d}{Display};
	\domain[x=5,y=-1.3,type=biddableDomain]{eo}{Enquiry operator};

	\connects[xshift=-2mm]{m}{rw}{RW!C1}
	\connects{m}{d}{AM!E3}
	\connects[label position=below,xshift=-2mm]{m}{eo}{EO!E5}

	\requirement[x=10,y=0]{r}{Answer rules};

	\refers{r}{rw}{C2}
	\constrains{r}{d}{Y4}
	\refers[label position=below]{r}{eo}{E5}

\end{tikzpicture}


\caption{{Commanded Information} {Problem Frame} (cf.~\cite{Jackson:2001}) \label{fig:ci}}
\end{figure}
\lstinputlisting{Figures/ci.tex}

\subsubsection{Simple Workpiece}
\begin{figure}[htbp]
\centering
\begin{tikzpicture}

	\domain[x=0,y=0,type=machine]{m}{Editor};
	\domain[x=5,y=.7,type=lexicalDomain]{w}{Workpieces};
	\domain[x=5,y=-.7,type=biddableDomain]{u}{User};

	\connects{m}{w}{E!E1\\ WP!Y2}
	\connects[label position=below]{m}{u}{U!E3}

	\requirement[x=10,y=0]{r}{Command effects};

	\constrains{r}{w}{Y4}
	\refers[label position=below]{r}{u}{E3}

\end{tikzpicture}


\caption{{Simple Workpiece} {Problem Frame} (cf.~\cite{Jackson:2001}) \label{fig:sw}}
\end{figure}
\lstinputlisting{Figures/sw.tex}

\subsubsection{Transformation}
\begin{figure}[htbp]
\centering
\begin{tikzpicture}

	\domain[x=0,y=0,type=machine]{m}{Transform machine};
	\domain[x=5,y=.7,type=lexicalDomain]{i}{Inputs};
	\domain[x=5,y=-.7,type=lexicalDomain]{o}{Outputs};

	\connects{m}{i}{I!Y1}
	\connects[label position=below]{m}{o}{TM!Y2}

	\requirement[x=10,y=0]{r}{IO relation};

	\constrains[label position=below]{r}{o}{Y4}
	\refers{r}{i}{Y3}

\end{tikzpicture}


\caption{{Transformation} {Problem Frame} (cf.~\cite{Jackson:2001}) \label{fig:t}}
\end{figure}
\lstinputlisting{Figures/t.tex}

\input{Bib/index}

\end{document}

